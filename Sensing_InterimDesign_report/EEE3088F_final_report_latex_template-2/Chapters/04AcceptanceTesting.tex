% ----------------------------------------------------
% Acceptance Testing
% ----------------------------------------------------
\documentclass[class=report,11pt,crop=false]{standalone}
\input{../Style/ChapterStyle.tex}
\begin{document}
% ----------------------------------------------------
\chapter{Acceptance Testing} \label{ch:atp}
\vspace{-1cm}
% ----------------------------------------------------

\section{Tests}
\begin{table}[h]
  \centering
    \caption{Subsystem Acceptance Tests}
    \label{tab:tests}
    \begin{tabular}{ >{\centering\arraybackslash}m{1cm}  m{4cm} m{5cm} m{5cm}}
      \hline
      \textbf{Test ID} & \textbf{Description} & \textbf{Testing Procedure}& \textbf{Pass/Fail Criteria} \\   
      \hline
      AT01 & Battery Voltage Check - see if subsystem battery provides adequate power (Tests R01, SP01)  & Use IR Emitter and IR Receiver circuit battery test points (TP1 and TP8) to measure battery voltage. & Battery voltage is within acceptable range (e.g., 3.7V for lithium-ion battery). \\
       \hline
      AT02 & Size check - see if subsystem fits given dimensions (Tests R02, SP02) & Measure physical dimensions of the pcb board & Should measure approximately 85mm width and 50mm height \\
       \hline
      AT03 & Cost check - check if subsystem meets the budget constraint(Tests R03, SP03)  & Upload the gerber files onto JLCPCB and generate instant an quote & Total cost should display less than \$30 \\
      \hline
      AT04 & Voltage Output Verification (Tests R04, SP04)  & Use output voltage test points (TP3, TP4, TP5) to measure output voltages at specified distances from a wall on the left, right and front. & Output voltages should increase when close to the wall and decrease when further away. \\
      \hline
      AT05 & Differentiation between left, right and front (Tests R05, SP05) & Similar to AT04, use output voltage test points (TP3, TP4, TP5) to measure output voltages at roughly 60mm from a wall on the left, right and front. & LED for each specific direction should light up accordingly \\
      \hline
      AT06 & Power Saving mode check(Tests R06, SP06)  & Vary duty cycle from 100\% to 50\% and time how long until the battery drains. Use a multi-meter to measure current using jumpers (JP10, JP9, JP8). &  Current through multi-meter should be roughly half at 50\% than at 100\% duty cycle. Battery drain time should be approximately double at 50\% duty cycle than at 100\% duty cycle  \\
      \hline
    \end{tabular}
\end{table}

% =====================================================
\section{Critical Analysis of Testing}
% This Section only needs to be completed for the final report

% In this section you are required to provide an in depth analysis of your testing and the outcomes of your tests. Reflect on what went wrong and why that is. What should you have done differently? Did your failure management help you, how should you have better designed for failure? How have you, if possible, overcome the failure?

\begin{table}[h]
  \begin{center}
    \caption{Subsystem acceptance test results}
    \label{tab:testresults}
    \begin{tabular}{ >{\centering\arraybackslash}m{1cm}  m{4cm} m{1cm}}
      \hline
      \textbf{Test ID} & \textbf{Description} & \textbf{Result}\\   
      \hline
      AT01 & Powers on & \\
       &  & \\
      \hline 
    \end{tabular}
  \end{center}
\end{table}

\subsection{AT01}
This worked, this did not work. I suspect that is because of x y and z.

% ----------------------------------------------------
\ifstandalone
\bibliography{../Bibliography/References.bib}
\fi
\end{document}
% ----------------------------------------------------