% ----------------------------------------------------
% Introduction
% ----------------------------------------------------
\documentclass[class=report,11pt,crop=false]{standalone}
\input{../Style/ChapterStyle.tex}
\begin{document}
% ----------------------------------------------------
\chapter{Introduction} \label{ch:introduction}
\vspace{-1cm}
% ----------------------------------------------------

\section{Problem Description}
The given task entails a project centered on the creation of a micro-mouse. A micro-mouse is essentially a maze solving robot which moves, detects and avoids obstacles in order to autonomously solve the maze.  \\ \\
\textbf{The Greater Project:} \\
The greater project involves building a fully functional micro-mouse robot capable of solving mazes autonomously. This includes designing and integrating various subsystems. This includes designing and integrating various subsystems such as the processor (controls various functions and movements, while communicating with each subsystem), motherboard module (responsible for connecting all the PCBs together, and is the base board that all other modules will slot onto.), power supply module (Supplies power to the entire system), and sensing module (responsible for detecting/sensing objects.) \\ \\
My specific task entails the designing and manufacturing the sensing subsystem that can reliably detect obstacles and walls in the robot's path. This module needs to interface with the motherboard and processor, providing the necessary information for the robot to make navigation decisions. 
%The project at hand involves the design of a sensing subsystem of a micro-mouses hardware which is capable of detecting the presence of a wall using infrared light, which is powered by a LiPo 800mAh 3.7V battery using components ordrdered from JLCPCB whilst adhering to a strict budget of \$30 per board.%
% =====================================================
\section{Scope and Limitations}
\textbf{Scope:} \\
 The sensing subsystem will involve detecting whether there is an obstacle in the front and on the sides of the robot, and have a switching means to save power when not in operation. The sensing subsystem must also be interfaced with the rest of the system, and make use of 3 LEDs, each one to demonstrate if a wall is being sensed on the left, right or straight ahead.  \\ \\
\textbf{Limitations:} \\
The sensing subsystem will be limited to making use of only infrared light to detect walls/objects. It will be interfaced with the STM32L476 processor. The subsystem will adhere to a strict budget of \$30 which may restrict development and the PCB and components are to be ordered exclusively from JLCPCB, which may impact component selection. The time constraint on the development of the project will restrict testing.

% =====================================================
\section{GitHub Link}
\url{https://github.com/sachenpather/EEE3088F_FinalReport_Group16}
% Make sure that the repo is public at the time of this submission.
% ----------------------------------------------------
\ifstandalone
\bibliography{../Bibliography/References.bib}
\fi
\end{document}
% ----------------------------------------------------